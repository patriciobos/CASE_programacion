\documentclass[aspectratio=169]{beamer}
%la opcion hangout es para complilar en modo imprimible
%\documentclass[hangout]{beamer}

\mode<presentation>
{
  \usetheme{Berkeley}
  \setbeamercovered{transparent}
  \setbeamertemplate{navigation symbols}{}
}

\usepackage[spanish]{babel}
\usepackage[utf8]{inputenc}
\usepackage{tikz}
\usepackage{textpos}
\usepackage{hyperref}
\usepackage{caption}
%\captionsetup[figure]{labelformat=empty}

%\usetikzlibrary{shapes,arrows}
\setbeamerfont{author}{size=\large}
\setbeamerfont{institute}{size=\normalsize\bfseries}
\setbeamerfont{title}{size=\Large\bfseries}
\setbeamerfont{subtitle}{size=\huge}

\definecolor{darkblue}{RGB}{51,51,179}
\setbeamercolor{bgcolor}{fg=white,bg=darkblue}

\title[Clase 1]{Programación de Microcontroladores - Clase 1}
\subtitle{Clase 1 - C para Sistemas Embebidos}
\author[]{Esp. Ing. Patricio Bos}
\institute[CESE-FIUBA]{Carrera de Especialización en Sistemas Embebidos - FIUBA}
\date{\today}

%\subtitle{Framework para aplicaciones de control de ambientes}
\titlegraphic{\includegraphics[width=5cm]{./imagenes/red.jpg}}


\subject{Programación de Microprocesadores. Carrera de Especialización en Sistemas Embebidos}
% This is only inserted into the PDF information catalog. Can be left
% out. 

\pgfdeclareimage[height=1.5cm]{university-logo}{./imagenes/logo-facu-inverso.png}
\logo{\pgfuseimage{university-logo}}


% If you wish to uncover everything in a step-wise fashion, uncomment
% the following command: 

\beamerdefaultoverlayspecification{<+->}
  
\begin{document}

%Captions sin el texto "Figure"
\setbeamertemplate{caption}{\raggedright\insertcaption\par}


%-------------------------------------------------%
%-------------------------------------------------%
% PORTADA
%-------------------------------------------------%
%-------------------------------------------------%

\begingroup
\makeatletter
\setlength{\hoffset}{-.5\beamer@sidebarwidth}
\makeatother
\begin{frame}[plain,noframenumbering]
\begin{center}
%\vspace{5px}
\hfill
    \begin{beamercolorbox}[center,dp=3ex,ht=10.25ex, wd=1\linewidth]{bgcolor}
        \Large\textbf{Programación de Microcontroladores}\\
        \vspace{5px}
        \Large\textbf{Clase 1}
    \end{beamercolorbox}
\hfill\hfill
\\
\vspace{5px}
\textbf{Carrera de Especialización en Sistemas Embebidos - FIUBA}\\
\vspace{10px}
\texttt{Esp. Ing. Patricio Bos}\\

\vfill
\tiny versión: 2017-02-8 rev 0.1 

\end{center}
\end{frame}
\endgroup

%-------------------------------------------------%
%-------------------------------------------------%

\begin{frame}{\textbf{Organización de la presentación}}
  \tableofcontents
  % You might wish to add the option [pausesections]
\end{frame}
%
%

%-------------------------------------------------%
%-------------------------------------------------%
\section{Introducción}
%-------------------------------------------------%
%-------------------------------------------------%


\subsection{Consideraciones del Curso}

\begin{frame}{Disclaimer}
	\begin{itemize}
		\item Las presentaciones de este curso está basadas en el material de la asignatura ``Sistemas Digitales'' de la UNQ a cargo del prof. Ing. José Juarez.
		\vspace{10px}
		\item Los ejemplo de código fueron recopilados y preparados por el Esp. Ing. Eric Pernia.
	\end{itemize}
\end{frame}

\begin{frame}{Condiciones de aprobación}
	\begin{itemize}
		\item 6 ejercicios prácticos de entrega obligatoria.  Plazo máximo de entrega 2 semanas desde la entrega del enunciado.
		\vspace{10px}
		\item Examen teórico escrito multiple-choice en la última clase.
  	\vspace{10px}
		\item 75\% de asistencia a las clases.
	\end{itemize}
\end{frame}

\begin{frame}{Bibliografía}
	\begin{itemize}
		\item The C programming languaje K\& R
		\vspace{10px}
		\item Expert C programming \- Deep C sectrets
  	\vspace{10px}
		\item The C puzzle book
	\end{itemize}
\end{frame}



\section{Repaso de C}

\subsection{Motivación para usar C}

\begin{frame}{Motivación para usar C}
	\begin{itemize}
		\item Lenguaje estructurado de ``mediano'' nivel (o assembler de alto nivel).
		\vspace{10px}
		\item Razonablemente fácil de aprender, comprender, programar y depurar.
    \vspace{10px}		  
		\item Acceso directo a las APIs de bajo nivel del HW (seguramente escritas en C).
    \vspace{10px}		
    \item Es procesador-independiente y no está atado a ningún microprocesador en particular.	
    \vspace{10px}
		%\item Reducción de tiempo de desarrollo mediante reutilización de código (bibliotecas).
		%\item Con buenas prácicas de programación se puede facilitar/mejorar legibilidad, mantenibilidad, portabilidad, etc...
		%\vspace{10px}
		\item Existencia de compiladores gratuitos para prácticamente todos los microprocesadores del mercado.
		\vspace{10px}
		\item Extenso soporte (bibliografía, cursos, artículos, foros, etc...).
		\vspace{10px}
	\end{itemize}
\end{frame}


\subsection{Consideraciones de performance}

\begin{frame}{Eficiencia y desempeño del lenguaje C}
  
\begin{minipage}[c]{1.0\linewidth}

	\begin{minipage}[c]{0.5\textwidth}
    \begin{itemize}
      \item La eficiencia más alta en cuanto a tamaño de código y velocidad de ejecución se logra en assembler.  Un compilador de C puede acercase a esta eficiencia si está optimizado para la arquitectura del uC.
      \vspace{10px}
      \item Comparado con assembler, en C se puede generar código más confiable, portable, mantenible y escalable.
      \vspace{10px}
      \item Comparado con otros lenguajes de alto nivel, en C se puede generar código más compacto y eficiente.
    \end{itemize}	
  \end{minipage}
	\hspace{.1\linewidth}
	\begin{minipage}[c]{0.4\linewidth}
		\begin{figure}[H]
			{\includegraphics[width=\textwidth]{./imagenes/CvsAssembly.png}}
		\end{figure}	  	  	
	\end{minipage}
\end{minipage}

\end{frame}  
  
\begin{frame}{Ejemplos de optimización}
  \begin{itemize}
    \item uC con arquitectura basada en acumuladores trabajan mejor con variables globales.
    \vspace{10px}
    \item uC con arquitectura basada en registros trabajan mejor con variables locales.
    \vspace{10px}
    \item Las variables locales permiten mejor modularidad y por lo tanto código más portable y reutilizable.
    \vspace{10px}
    \item Si el desempeño de una sección de código es crítica se puede hacer en assembler.
  \end{itemize}
\end{frame}


\begin{frame}
control de flujo
directivas precompilador
tipos
prefijos sufijos
modificadores 
operadores aritméticos
operadores de bit

\end{frame}

\section{C para sistemas embebidos}

\begin{frame}{}
	\begin{itemize}
		\item
		\vspace{5px}
		\item
	\end{itemize}
\end{frame}


\begin{frame}{\textbf{Tipos de variables}}
	\begin{itemize}
		\item \textbf{\texttt{global}}: Almacenamiento global y visibilidad global.
		\vspace{5px}
		\item \textbf{\texttt{static}}: Almacenamiento global y visibilidad local.
		\vspace{5px}
		\item \textbf{\texttt{local}}: Almacenamiento local y visibilidad local.
	\end{itemize}
\end{frame}

\begin{frame}{Modificadores de clase de almacenamiento}
	\begin{itemize}
		\item static
		\vspace{5px}
		\item volatile
		\vspace{5px}
		\item const
	\end{itemize}
\end{frame}

\subsection{Control de flujo}



%%-------------------------------------------------%
%\subsection[Topología]{Topología de la red}
%%-------------------------------------------------%
%
%\begin{frame}{Topología de la Red}{Estrella o punto a punto}
%
%
%
%\begin{frame}{Topología Punto a punto}{Árbol de Cluster}
%\begin{minipage}[c]{1.0\linewidth}
%\begin{minipage}[c]{0.45\linewidth}
%		\begin{itemize}
%			\item Mayoría de FFDs.
%			\vspace{10px}
%			\item 1 \textit{overall PAN coordinator}.
%			\vspace{10px}
%			\item RFDs al final de una rama.
%			\vspace{10px}
%			\item Aumenta el área de covertura.
%			\vspace{10px}
%			\item Aumenta la latencia de la red.
%		\end{itemize}	
%	\end{minipage}
%	\hspace{-15px}
%	\begin{minipage}[c]{0.65\linewidth}
%		\begin{figure}[H]
%			{\includegraphics[width=.8\textwidth]{./imagenes/cluster}}
%		\end{figure}	  	  	
%	\end{minipage}
%\end{minipage}
%\end{frame}
%
%%-------------------------------------------------%
%\subsection[Arquitectura]{Arquitectura del estándar}
%%-------------------------------------------------%
%
%\begin{frame}{Arquitectura del estándar}
%
%\begin{minipage}[c]{1.0\linewidth}
%	\begin{minipage}[c]{0.6\linewidth}
%		\begin{itemize}
%			\item MAC Sublayer
%			\begin{itemize}
%				\item Beacon management
%				\item Channel access
%				\item GTSs management
%				\item Frame validation, ACKs
%				\item Asociación y desasociación de dispositivos
%			\end{itemize}
%			\vspace{10px}
%			\item Physical Layer (PHY):
%			\begin{itemize}
%				\item Activación/Desactivación de RF
%				\item ED, LQI, Clear Channel Assessment (CCA)
%				\item Channel selection
%				\item Tx y Rx de paquetes a través del medio físico
%			\end{itemize}
%			\vspace{10px}
%		\end{itemize}	
%	  \end{minipage}
%	  \begin{minipage}[c]{0.35\linewidth}
%		\begin{figure}[H]
%			{\includegraphics[width=.6\textwidth]{./imagenes/arquitectura}}
%		\end{figure}	  	  	
%	  \end{minipage}
%\end{minipage}
%
%\end{frame}
%
%
%\begin{frame}[t]{MAC: Beacons, Supertramas y GTSs}
%\begin{minipage}[c]{1.0\linewidth}
%\begin{minipage}[c]{0.35\linewidth}
%		\begin{itemize}
%			\item 16 time slots.
%			\vspace{5px}
%			\item Contention Access Period (CAP)\\ \textbf{con CSMA/CA}.
%			\vspace{5px}
%			\item Contention Free Period (CFP) para los GTSs, \textbf{sin CSMA/CA}.
%			\vspace{5px}
%			\item Los GTSs son opcionales y reducen el CAP.
%			\vspace{5px}
%			\item Tiempo inactivo $\rightarrow$ modo bajo consumo.
%		\end{itemize}	
%	\end{minipage}
%%	\hspace{-5px}
%	\begin{minipage}[c]{0.65\linewidth}
%		\begin{figure}[H]
%			\includegraphics[width=1\linewidth]{./imagenes/superFrame}\\
%		\end{figure}	  	  	
%	\end{minipage}
%\end{minipage}
%%\vspace{10px}
%%	\begin{figure}[H]
%%		\includegraphics[height=.35\textheight]{./imagenes/superFrame}\\
%%	\end{figure}
%
%\end{frame}
%
%%--------------------------------------------------------------------%
%\subsection[Transferencia de datos]{Modelo de Transferencia de datos}
%%--------------------------------------------------------------------%
%
%
%\begin{frame}[t]{Transferencia de Datos}
%\textbf{Con beacon}
%%\vspace{10px}
%\begin{columns}[t]
%	\begin{column}{.5\textwidth}
%		\begin{minipage}[t][0.7\textheight][s]{\columnwidth}
%			\begin{figure}[H]
%				\includegraphics[height=.65\textheight]{./imagenes/dev-coord-beacon.jpg}
%				\vfill
%				\vspace{10px}
%				\caption{Device $\rightarrow$ Coordinator}
%			\end{figure}
%		\end{minipage}
%	\end{column}
%	\begin{column}{.5\textwidth}
%		\begin{minipage}[t][0.7\textheight][s]{\columnwidth}
%			\begin{figure}[H]
%				\includegraphics[height=.7\textheight]{./imagenes/coord-dev-beacon.jpg}
%				\vfill
%				\caption{Coordinator $\rightarrow$ Device}
%			\end{figure}	 		
%		\end{minipage}
%	\end{column}
%\end{columns}
%\end{frame}
%
%\begin{frame}[t]{Transferencia de Datos}
%\textbf{Sin beacon}
%%\vspace{10px}
%\begin{columns}[t]
%	\begin{column}{.5\textwidth}
%		\begin{minipage}[t][0.7\textheight][s]{\columnwidth}
%			\begin{figure}[H]
%				\includegraphics[height=.45\textheight]{./imagenes/dev-coord-sinbeacon.jpg}
%				\vfill
%				\vspace{10px}
%				\caption{Device $\rightarrow$ Coordinator}
%			\end{figure}
%		\end{minipage}
%	\end{column}
%	\begin{column}{.5\textwidth}
%		\begin{minipage}[t][0.7\textheight][s]{\columnwidth}
%			\begin{figure}[H]
%				\includegraphics[height=.5\textheight]{./imagenes/coord-dev-sinbeacon.jpg}
%				\vfill
%				\caption{Coordinator $\rightarrow$ Device}
%			\end{figure}	 		
%		\end{minipage}
%	\end{column}
%\end{columns} 	  	
%\end{frame}
%
%
%%%--------------------------------------------------------------------%
%%\subsection[CSMA/CA]{CSMA/CA}
%%%--------------------------------------------------------------------%
%%
%%\begin{frame}[T]{Carrier Sense multiple Access with Collision Avoidance}
%%
%%Slotted CSMA/CA vs Unslotted CSMA/CA
%%
%%\vspace{10px}
%%
%%		\begin{figure}[H]
%%			\includegraphics[height=.8\textheight]{./imagenes/CSMA-CA}\\
%%%			\vspace{15px}
%%%			\includegraphics[height=.3\textheight]{./imagenes/unsolotted.jpg}
%%		\end{figure}
%%\end{frame}
%
%
%%--------------------------------------------------------------------%
%\subsection[Tramas]{Estructura de las Tramas}
%%--------------------------------------------------------------------%
%
%\begin{frame}{Estructura de las Tramas}
%Se definen 4 tipos de trama MAC:
%\vspace{5px}
%	\begin{itemize}
%		\item Beacon
%		\vspace{5px}
%		\item Data
%		\vspace{5px}
%		\item Acknowledgement
%		\vspace{5px}
%		\item MAC Command
%		\vspace{5px}
%	\end{itemize}
%	\begin{figure}[H]
%		\includegraphics[width=.8\textwidth]{./imagenes/frameStructure}
%	\end{figure}	 	
%\end{frame}
%
%\begin{frame}[t]{Tramas}
%Tipo de Frame: \textbf{Beacon}
%\vspace{10px}
%	\begin{figure}[H]
%		\includegraphics[width=1\textwidth]{./imagenes/beacon.jpg}
%	\end{figure}	  	  	
%\end{frame}
%
%\begin{frame}[t]{Tramas}
%Tipo de Frame: \textbf{Data}
%\vspace{10px}
%	\begin{figure}[H]
%		\includegraphics[width=1\textwidth]{./imagenes/data.jpg}
%	\end{figure}	  	  	
%\end{frame}
%
%
%\begin{frame}[t]{Tramas}
%Tipo de Frame: \textbf{Acknowledgement} (Ack)
%\vspace{10px}
%	\begin{figure}[H]
%	\centering
%		\includegraphics[width=.7\textwidth]{./imagenes/ack.jpg}
%	\end{figure}	  	  	
%\end{frame}
%
%\begin{frame}[t]{Tramas}
%Tipo de Frame: \textbf{MAC Command}
%\vspace{10px}
%	\begin{figure}[H]
%		\includegraphics[width=1\textwidth]{./imagenes/maccommand.jpg}
%	\end{figure}	  	  	
%\end{frame}
%
%
%\begin{frame}[t]{Tramas}
%Frame Control Field
%	\begin{figure}[H]
%	\centering
%		\includegraphics[height=.3\textheight]{./imagenes/FCF.jpg}\\
%		\vspace{20px}
%		\includegraphics[height=.3\textheight]{./imagenes/frametype.jpg}
%		\hspace{20px} 
%		\includegraphics[height=.3\textheight]{./imagenes/addressingmode.jpg}
%	\end{figure} 	
%\end{frame}
%
%
%%--------------------------------------------------------------------%
%\subsection[Modulación]{Modulación}
%%--------------------------------------------------------------------%
%
%\begin{frame}{Modulación}
%\begin{minipage}[c]{1.0\linewidth}
%\begin{figure}[H]
%	\includegraphics[height=1\textheight]{./imagenes/modulaciones.jpg}
%		\end{figure}	
%\end{minipage}
%\end{frame}
%
%%-------------------------------------------------%
%%-------------------------------------------------%
%\section{Mote LSE}
%%-------------------------------------------------%
%%-------------------------------------------------%
%
%%-------------------------------------------------%
%\subsection[Mote]{Mote}
%%-------------------------------------------------%
%\begin{frame}{Nodo Mote LSE-FIUBA} 
%
%\begin{minipage}[c]{1.0\linewidth}
%	\begin{minipage}[c]{0.6\linewidth}
%		\begin{itemize}
%			\item Nodo Mote desarrollado en el LSE-FIUBA
%			\vspace{15px}
%			\begin{itemize}
%				\item LPC1343 ARM Cortex-M3 @72MHz
%				\vspace{5px}
%				\item Transceptor TI-2520
%				\vspace{5px}
%				\item Extensor de rango TI-2591
%				\vspace{5px}
%				\item 3 Pulsadores
%				\vspace{5px}
%				\item 3 leds
%				\vspace{5px}
%				\item Antena y balun en microstrip
%			\end{itemize}
%			\vspace{10px}
%		\end{itemize}
%	\end{minipage}
%	\begin{minipage}[c]{0.35\linewidth}
%		\begin{figure}[H]
%			\vspace{35px}
%			\includegraphics[width=1\textwidth]{./imagenes/mote.jpg}
%			\\
%			\vspace{10px}
%			\includegraphics[width=1\textwidth]{./imagenes/motePCB}
%			\label{Mote LSE}
%			%\caption{Mote LSE}
%		\end{figure}	  	  	
%	\end{minipage}
%\end{minipage}
%\end{frame}
%
%
%\begin{frame}{Nodo Mote LSE-FIUBA}{Circuito Esquemático}
%	\begin{figure}[H]
%		\includegraphics[height=.95\textheight]{./imagenes/moteSchema}
%	\end{figure}	  	  	
%\end{frame}
%
%
%
%%-------------------------------------------------%
%\subsection[TI CC2520]{TI CC2520}
%%-------------------------------------------------%
%\begin{frame}{Transceptor DSSS TI CC2520} 
%\begin{minipage}[c]{1.0\linewidth}
%	\begin{minipage}[c]{0.7\linewidth}
%		\begin{itemize}
%			\vspace{5px}
%			\item 2394-2507 MHz	
%%			\item VCO, LNA, PA y filtros On-chip	
%%			\item 3 modos de potencia flexibles para consumo reducido
%			\vspace{10px}
%			\item Muy bajo consumo de corriente
%			\begin{itemize}
%				\item RX: 18.5 - 22.3 mA.
%				\item TX: 25.8 - 33.6 mA.
%			\end{itemize}
%			\vspace{10px}
%			\item Interfaz de usuario
%			\begin{itemize}
%				\item SPI
%				\item 6 GPIOs
%% 				\item Generador de interupciones
%				\item Respuestas automáticas a diferentes eventos
%				\item Modo de Packet Sniffer embebido
%			\end{itemize}
%			\vspace{10px}
%		\end{itemize}
%	\end{minipage}
%	\begin{minipage}[c]{0.25\linewidth}
%		\begin{figure}[H]
%			\includegraphics[width=1\textwidth]{./imagenes/cc2520.jpg}
%			\label{cc2520}
%			%\caption{cc2520}
%		\end{figure}	  	  	
%	\end{minipage}
%\end{minipage}
%\end{frame}
%
%
%\begin{frame}{Soporte por Hardware a 802.15.4 MAC} 
%\begin{minipage}[c]{1.0\linewidth}
%	\begin{minipage}[c]{0.70\linewidth}
%		\begin{itemize}
%			\vspace{5px}
%			\item Generador automático de preámbulo	
%			\vspace{5px}
%			\item Inserción y detección de palabra de sincronización	
%			\vspace{5px}
%			\item CRC-16 en el MAC payload
%			\vspace{5px}
%			\item Frame Filtering
%			\vspace{5px}
%			\item Ack automático
%			\vspace{5px}
%			\item Clear Channel Assessment (CCA)
%			\vspace{5px}
%			\item Energy Detection (ED)
%			\vspace{5px}
%			\item Link Quality Indication (LQI)
%%			\vspace{5px}
%%			\item Seguridad MAC automática (CTR, CBC-MAC,CCM)
%		\end{itemize}
%	\end{minipage}
%	\begin{minipage}[c]{0.25\linewidth}
%		\begin{figure}[H]
%			\includegraphics[width=1\textwidth]{./imagenes/cc2520.jpg}
%		\end{figure}	  	  	
%	\end{minipage}
%\end{minipage}
%\end{frame}
%
%\begin{frame}{Circuito de Aplicación Típico}	
%	\begin{figure}[H]
%		\includegraphics[height=1\textheight]{./imagenes/applicationcircuit.jpg}
%	\end{figure}	
%\end{frame}
%
%\begin{frame}{Diagrama Funcional}
%	\begin{figure}[H]
%		\includegraphics[height=1\textheight]{./imagenes/diagrama.jpg}
%	\end{figure}	
%\end{frame}
%
%\begin{frame}{Procesamiento de tramas: Tx}
%	\begin{figure}[H]
%		\includegraphics[height=.95\textheight]{./imagenes/txfifo.jpg}
%	\end{figure}	
%\end{frame}
%
%\begin{frame}{Procesamiento de tramas: Rx filtering}
%	\begin{figure}[H]
%		\includegraphics[height=.95\textheight]{./imagenes/filtering.jpg}
%	\end{figure}	
%\end{frame}
%
%
%\begin{frame}{Procesamiento de tramas: Rx matching}
%	\begin{figure}[H]
%		\includegraphics[height=.95\textheight]{./imagenes/matching.jpg}
%	\end{figure}	
%\end{frame}
%
%
%%-------------------------------------------------%
%%-------------------------------------------------%
%\section{Referencias}
%%-------------------------------------------------%
%%-------------------------------------------------%
%
%\begin{frame}[c]{Referencias}
%
%%\Large{Referencias}
%\vspace{20px}
%\begin{itemize}
%	\item<.-> \href{http://ecee.colorado.edu/~liue/teaching/comm_standards/2015S_zigbee/802.15.4-2011.pdf}{Estándar IEEE 802.15.4:2011}
%	\vspace{5px}	
%	\item<.-> \href{http://www.ieee802.org/15/pub/TG4.html}{IEEE 802.15 - Task Group 4 - Home Page}
%	\vspace{5px}
%	\item<.-> \href{	https://standards.ieee.org/about/get/802/802.15.html}{IEEE Get Program}
%	\vspace{5px}
%	\item<.-> \href{http://www.nxp.com/documents/data_sheet/LPC1311_13_42_43.pdf}{LPC1343 - Datasheet}
%	\vspace{5px}
%	\item<.-> \href{http://www.nxp.com/documents/user_manual/UM10375.pdf}{LPC1343 - User Manual}
%	\vspace{5px}
%	\item<.-> \href{http://www.ti.com/product/CC2520/technicaldocuments}{Texas Instrument CC2520 -  Technical Documents}
%	\vspace{5px}
%	\item<.-> \href{http://www.ti.com/lit/an/swru120b/swru120b.pdf}{Texas Instrument Design Note - 2.4 GHz Inverted F Antenna}
%\end{itemize}
%\end{frame}

%\section*{cartón de gracias}

\begingroup
\makeatletter
\setlength{\hoffset}{-.5\beamer@sidebarwidth}
\makeatother
\begin{frame}[plain,noframenumbering]
\begin{center}
%\vspace{5px}
\hfill
    \begin{beamercolorbox}[center,dp=3ex,ht=10.25ex, wd=1\linewidth]{bgcolor}
        \Large\textbf{Programación de Microcontroladores}\\
        \vspace{5px}
        \Large\textbf{Clase 1}
    \end{beamercolorbox}
\hfill\hfill
\\
\vspace{5px}
\textbf{Carrera de Especialización en Sistemas Embebidos - FIUBA}\\
\vspace{10px}
Esp. Ing. Patricio Bos\\ 
\texttt{pbos@fi.uba.ar}\\


\vspace{10px}

%\begin{figure}[H]
%	\includegraphics[width=.3\textwidth]{./imagenes/red.jpg}
%\end{figure}	

\vspace{5px}
\vfill
\tiny versión: 2016-08-20 rev 0.1 
 	  	
\end{center}
\end{frame}
\endgroup

\end{document}
